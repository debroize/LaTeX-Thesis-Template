%%%%%%%%%%%%%%%%%%%%%%%%%%%%%%%%%%%%%%%
%%%% Freeze + F1-Loss
%%%%%%%%%%%%%%%%%%%%%%%%%%%%%%%%%%%%%%%
% \subsubsection{Freezeless + F1-Loss}
% ~\label{sec:MinorTests:sub:FreezlessF1}
% \subsection{Freezeless}
% ~\label{sec:MinorTests:sub:Freezeless}
% %%%
% %%%
% In this section, two small tests are presented.
% Firstly, a model that was trained for 80 epochs on iSet 2 with InceptionResNetV2 is further trained for 20 epochs, but during these epochs, the base model is opened for weight adaptation (\emph{unfreezed}).
% The training curves (\prettyref{fig:Ap_Freezeless}) show, that this is an approach, that can improve the model further, but the gains in accuracy are that small, that it does not justify the additional training time. 
% Also the increasing validation loss indicates that the risk of overfitting is rising.

% \begin{figure}[H]
%     \centering
%     \includegraphics[width=\textwidth]{figures/appendix/minor_tests/longrun_iset2_inc_freezeless.pdf}
%     % \subfigure[]{\label{fig:Ap_CollComp_Smpl_FinRes}\includegraphics[width=\textwidth]{figures/appendix/minor_tests/collcomp/longrun_collages_pivot.pdf}}
    
%     \caption{Freezless post-training of InceptionResNetV2 on iSet 2 classification.
%             }
%     \space\label{fig:Ap_Freezeless}
% \end{figure}%


\subsection{F1-Loss}
~\label{sec:MinorTests:sub:F1}
In this small test, it is investigated, if the replacement of the binary cross-entropy loss with the F1-score loss can improve the classification performance.
The test is done with the MobileNetV2-SGD-64 configuration on iSet 2 and 8. 
For comparison, the run with the binary cross-entropy loss is also presented in the plots of \prettyref{fig:Ap_F1}.
The results show, that the F1-score loss is not able to improve the model performance.
On iSet 8, the F1-score loss even leads to a lower accuracy.


\begin{figure}[H]
    \centering
    \includegraphics[width=\textwidth]{figures/appendix/minor_tests/longrun_mob_f1.pdf}
   
    \caption{Comparison between binary cross-entropy (BCE) and F1-score loss training on MobileNetV2-SGD-64 on iSet 2 and 8.
            }
    \space\label{fig:Ap_F1}
\end{figure}%



The F1-score loss was implemented by:

\begin{python}
    @tf.function
    def f1_loss(y_true, y_pred):
        y_true = tf.cast(y_true, tf.float32)
        y_pred = tf.cast(y_pred, tf.float32)
        
        # precision and recall
        tp = tf.cast(tf.reduce_sum(y_pred * y_true), tf.float32)
        fp = tf.cast(tf.reduce_sum(y_pred * (1 - y_true)), tf.float32)
        fn = tf.cast(tf.reduce_sum((1 - y_pred) * y_true), tf.float32)
        
        epsilon = 1e-7
        precision = tf.math.divide_no_nan(tp, tp + fp + epsilon)
        recall = tf.math.divide_no_nan(tp, tp + fn + epsilon)

        # Return the F1 score
        f1_score = 2 * tf.math.divide_no_nan((precision * recall), 
                                            (precision + recall) + epsilon)
        return (1-f1_score)
\end{python}


\newpage
\subsection{Freezeless}
~\label{sec:MinorTests:sub:Freezeless}
%%%
%%%
In this test, a model that was trained for 80 epochs on iSet 2 with InceptionResNetV2 is further trained for 20 epochs, but during these epochs, the base model is opened for weight adaptation (\emph{unfreezed}).
The training curves (\prettyref{fig:Ap_Freezeless}) show, that this is an approach, that can improve the model further, but the gains in accuracy are that small, that it does not justify the additional training time. 
Also the increasing validation loss indicates that the risk of overfitting is rising.

\begin{figure}[H]
    \centering
    \includegraphics[width=\textwidth]{figures/appendix/minor_tests/longrun_iset2_inc_freezeless.pdf}
    % \subfigure[]{\label{fig:Ap_CollComp_Smpl_FinRes}\includegraphics[width=\textwidth]{figures/appendix/minor_tests/collcomp/longrun_collages_pivot.pdf}}
    
    \caption{Freezless post-training of InceptionResNetV2 on iSet 2 classification.
            }
    \space\label{fig:Ap_Freezeless}
\end{figure}%




% This is a comparison between different image inputs for the classification task. 
% Image types are: Standard images (``Std''), Golden Ratio collages (``GoRa''), Transect collages (``TrCol'') and Plot collages (``PlCol'').
% Sample images of TrCol and PlCol in \prettyref{fig:Ap_CollComp_Smpl}.

% %% Sample Images
% \begin{multicols}{2}
%     \begin{minipage}{\columnwidth}
%         \begin{figure}[H]
%             \vspace{2.1\baselineskip}
%             \centering
%             % TrCol + PlCol
%             \subfigure[TrCol]{\label{fig:Ap_CollComp_Smpl_TrCol}\includegraphics[width=0.493\textwidth]{figures/appendix/minor_tests/collcomp/sampleimg/Collage_Pl_125_Tr_30.jpg}}
%             \subfigure[PlCol]{\label{fig:Ap_CollComp_Smpl_PlCol}\includegraphics[width=0.493\textwidth]{figures/appendix/minor_tests/collcomp/sampleimg/Collage_Pl_125.jpg}}

%             \vspace{1.78\baselineskip}
%             \caption{Sample images for different image inputs.
%                     }
%             \space\label{fig:Ap_CollComp_Smpl}
%         \end{figure}%
%     \end{minipage}

%     \begin{minipage}{\columnwidth}
%         \begin{figure}[H]
%             \centering
%             \subfigure[]{\label{fig:Ap_CollComp_Smpl_FinRes}\includegraphics[width=\textwidth]{figures/appendix/minor_tests/collcomp/longrun_collages_pivot.pdf}}
            
%             \caption{Sample images for different image inputs.
%                     }
%             \space\label{fig:Ap_CollComp_FinRes}
%         \end{figure}%
%     \end{minipage}
% \end{multicols}


% Each training is done with the SGD optimizer and batch sizes of 64 (Std + GoRa) and 12 (TrCol + PlCol) over 60 epochs.
% The final accuracy is visualized in \prettyref{fig:Ap_CollComp_FinRes} and the training curves are shown in \prettyref{fig:Ap_CollComp_TC}.


% %% Training Curves
% \begin{figure}[H]
%     \centering
%     %% Classification, iSet 2 
%     \subfigure[VGG16]{\label{fig:Ap_CollComp_TC_VGG}\includegraphics[width=0.9\textwidth]{figures/appendix/minor_tests/collcomp/longrun_imgtype_vgg.pdf}}
%     \subfigure[InceptionResNetV2]{\label{fig:Ap_CollComp_TC_Inc}\includegraphics[width=0.9\textwidth]{figures/appendix/minor_tests/collcomp/longrun_imgtype_inc.pdf}}
%     \subfigure[MobileNetV2]{\label{fig:Ap_CollComp_TC_Mob}\includegraphics[width=0.9\textwidth]{figures/appendix/minor_tests/collcomp/longrun_imgtype_mob.pdf}}
%     % \subfigure[PlCol]{\label{fig:Ap_CollComp_TC_PlCol}\includegraphics[width=\textwidth]{figures/appendix/minor_tests/collcomp/longrun_plotcol.pdf}}

%     \caption{Training curves, comparison between image inputs.
%             }
%     \space\label{fig:Ap_CollComp_TC}
% \end{figure}%

