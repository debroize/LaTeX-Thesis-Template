%%%%%%%%%%%%%%%%%%%%
%%% Figures
%%%%%%%%%%%%%%%%%%%%
%%%%% Sample Figure

\begin{figure}[ht]
    \begin{center}
      \includegraphics[width=\textwidth]{figures/\dots.png} % Path to figure
    \caption{\lipsum[3] % Caption text. optional: \caption[<list of figures caption>]{<in-text-caption>}
             }
    \space\label{fig:Sample_Figure}
    \end{center}
\end{figure} 

Cite figure~\prettyref{fig:Sample_Figure}

%%%%% Multi-Image Figure
\begin{figure}[htbp]
  \centering
  \subfigure[Text a]{\label{fig:Subfig_a}\includegraphics[width=.2666\textwidth]{figures/sample_images/P6102278.JPG}} % 4:3 image
  \subfigure[Text b]{\label{fig:Subfig_b}\includegraphics[width=.20\textwidth, angle=90]{figures/sample_images/P6102287.JPG}} % 3:4 image
  \subfigure[Text c]{\label{fig:Subfig_c}\includegraphics[width=.2666\textwidth]{figures/sample_images/P6102290.JPG}} % 4:3 image
  \subfigure[Text d]{\label{fig:Subfig_d}\includegraphics[width=.2666\textwidth]{figures/sample_images/P6102300.JPG}} % 4:3 image

  % \caption{Image type examples for (a) Stand, (b) Transect [top], (c) Transect [side] and (d) Floor}
  \caption{Image type examples}
  \space\label{fig:Main_Figure_Caption}
\end{figure}%

Cite subfigure \subfigref{fig:Main_Figure_Caption}{fig:Subfig_a}


%%%%%%%%%%%%%%%%%%%%
%%% Tables
%%%%%%%%%%%%%%%%%%%%
%%% Small Table (centered)
\begin{table}[ht]
  \captionsetup{justification=centering,margin=2cm, singlelinecheck=false, format=plain}
  \centering
  \begin{tabular}{@{}lllll@{}}
  \toprule
   & \textbf{WoodyShrub} & \textbf{FineShrub} & \textbf{MossHerb} & \textbf{Herb} \\ \midrule
  \textbf{R²}   & -0.999 & -0.175 & 0.203 & -1.835 \\
  \textbf{RMSE} & 5391   & 1307   & 844   & 1384   \\
  \textbf{MAE}  & 2150   & 512    & 486   & 764    \\
  \textbf{STD}  & 4944   & 1202   & 690   & 1154   \\ \bottomrule
  \end{tabular}
  \caption{Biomass, quantification prediction metrics, iSet 1.
          \newline
          RMSE: root mean squared error, MAE: mean absolute error, STD: standard deviation.
          }
  \space\label{tab:pred_metrics:met_iSet1_bm}
\end{table}


%%% Table over full page width
\begin{table}[ht]
  \caption{One-hot encoding example for each categorical target. The lookup table for encoded vectors is shown in \tabref{tab:OneHot_LUT}.}
  \begin{tabularx}{\textwidth}{@{}lcX@{}} 
      \toprule
      \textbf{Target}            & \textbf{Label}    & \textbf{One-Hot Encoded} \\ \midrule
      \textbf{Litter-Type}       & bl-ln             & [1, 0, 1, 0, 0] \\% \midrule
      \textbf{Understory-Type}   & shrub-broadleaf   & [0, 1, 0, 0, 0, 0, 0] \\% \midrule
      \textbf{Tree-Species}      & beech-oak-pine    & [1, 1, 0, 0, 1, 0, 0, 0, 0, 0, 0, 0] \\ \bottomrule
  \end{tabularx}
  \space\label{tab:OneHot_Cat}
\end{table}


%%% Multicolum + Multirow Table
\begin{table}[htbp]
  \caption{Analyzed parameters/targets, separated by sampling method (Plot-wise, Transect-wise) and numerical feature type (Categorical [discrete], Quantitative [continuous]).}
  \centering
  \begin{tabularx}{\textwidth}{lXX}
      \toprule
      & \multicolumn{2}{c}{\textbf{Feature}} \\ 
      & \multicolumn{1}{c}{Plot-wise} & \multicolumn{1}{c}{Transect-wise} \\
      \midrule
      \multirow{3}{*}{\textbf{Categorical}}  & \hfil Litter Type       &  \\ 
                                              & \hfil Understory Type   &  \\
                                              & \hfil Tree Species      &  \\
      \midrule
      \multirow{3}{*}{\textbf{Quantitative}} & \hfil Coverages         & \hfil Fine Woody Debris  \\ 
                                              & \hfil Biomass           & \hfil              \\
      \bottomrule
  \end{tabularx}
  \space\label{tab:Data_FeatureList} 
\end{table}

%% Teble to whole vertical page
% \begin{landscape}
%   %% insert table here
% \end{landscape}


Cite any table \tabref{tab:Data_FeatureList}


%%%%%%%%%%%%%%%%%%%%
%%% Equations
%%%%%%%%%%%%%%%%%%%%
%%% Simple Equation
\begin{equation}
  Biomass = \beta 1 * {({RCD}^2 * Height)}^{\beta 2}  
\end{equation}

%%% Equation with Case Bracket
\begin{equation}
  Loss_{bce} = -\sum_{i=1}^{n}\log(q_{i})
  \hspace{1cm}
  \text{for } \hspace{0.3cm} q_{i} = 
            \begin{cases} 
                            pred_{i}   & \text{if } target_{i} = 1 \\
                            1-pred_{i} & \text{if } target_{i} = 0
            \end{cases}
\end{equation}

%%%%%%%%%%%%%%%%%%%%
%%% Equations
%%%%%%%%%%%%%%%%%%%%
%%% Code Block
\begin{python}
  def tensor_load_image(path_file, size):
      img = tf.io.read_file(path_file)
      img = tf.image.decode_image(img, channels=3, dtype=tf.float32)
      img = tf.image.resize(img, (size, size))
      return img
  \end{python}


%%%%%%%%%%%%%%%%%%%%
%%% Citations
%%%%%%%%%%%%%%%%%%%%
\space\cite{Bolte2006} % regular
\citep{Keane2015} % parenthetical


%%%%%%%%%%%%%%%%%%%%
%%% List
%%%%%%%%%%%%%%%%%%%%
\begin{enumerate}
  \item Element 1
  \item Element 2
  \item Element 3
\end{enumerate}